% @author Thomas Prest
% @author Pierre-Alain Fouque
% @author Jeffrey Hoffstein
% @author Paul Kirchner
% @author Vadim Lyubashevsky
% @author Thomas Pornin
% @author Thomas Ricosset
% @author Gregor Seiler
% @author William Whyte
% @author Zhenfei Zhang
% @copyright The authors

\section{Notations}\label{sec:spec:notations}

\paragraph{Cryptographic parameters.} For a cryptographic signature scheme, $\lambda$ denotes its security level and $\queries$ the maximal number of signing queries. Following \cite{NIST}, we assume $\queries = 2^{64}$.

\paragraph{Matrices, vectors and scalars.} Matrices will usually be in bold uppercase (e.g. $\matB$), vectors in bold lowercase (e.g. $\vecv$) and scalars -- which include polynomials -- in italic (e.g. $s$). We use the row convention for vectors. The transpose of a matrix $\matB$ may be noted $\matB^\t$. It is to be noted that for a polynomial $f$, we do \emph{not} use $f'$ to denote its derivative in this document.

\paragraph{Quotient rings.} For $q \in \bN^\star$, we denote by $\bZ_q$ the quotient ring $\bZ/q\bZ$. In \falcon, our integer modulus $q = 12289$ is prime so $\bZ_q$ is also a finite field. We denote by $\bZ_q^\times$ the group of invertible elements of $\bZ_q$, and by $\varphi$ Euler's totient function: $\varphi(q) = |\bZ_q^\times| = q - 1 = 3 \cdot 2^{12}$ since $q$ is prime.

\paragraph{Number fields.} \falcon uses a polynomial modulus $\phi = x^n+1$ (for $n = 2^\kappa$). It is a monic polynomial of $\bZ[x]$, irreducible in $\bQ[x]$ and with distinct roots over $\bC$.

Let $a =\sum_{i=0}^{n-1} a_i x^i$ and $b =\sum_{i=0}^{n-1} b_i x^i$ be arbitrary elements of the number field $\cQ = \bQ[x]/(\phi)$.
% \begin{itemize}
%  \item
 We note $\adj{a}$ and call (Hermitian) adjoint of $a$ the unique element of $\cQ$ such that for any root $\zeta$ of $\phi$, $\adj{a}(\zeta) = \overline{a(\zeta)}$, where $\overline{\cdot}$ is the usual complex conjugation over $\bC$. For $\phi = x^n+1$, the Hermitian adjoint $\adj a$ can be expressed simply:
%  \begin{itemize}
%  \item \emph{Binary case.} If $\phi = x^n+1$ with $n = 2^\kappa$ a power of $2$, then
 \begin{equation}
 \adj a = a_0 - \sum_{i=1}^{n-1} a_{i} x^{n-i}
 \end{equation}
%  \item \emph{Ternary case.} If $\phi = x^n - x^{n/2} + 1$ with $n = 3 \cdot 2^\kappa$, then
% \begin{equation}
% \adj a = a_0 + \sum_{i=1}^{n-1} a_{i} (x^{n/2-i} - x^{n-i})
% \end{equation}
% \end{itemize}

We extend this definition to vectors and matrices: the adjoint $\adj\matB$of a matrix $\matB \in \cQ^{n\times m}$ (resp. a vector $\vecv$) is the component-wise adjoint of the transpose of $\matB$ (resp. $\vecv$):
\begin{equation}
\matB = \twotwo{a}{b}{c}{d} \quad \Leftrightarrow \quad \adj \matB = \twotwo{\adj a}{\adj c}{\adj b}{\adj d}
\end{equation}
%  \item

\paragraph{Inner product.} The inner product $\inner{\cdot}{\cdot}$ over $\cQ$ and its associated norm $\|\cdot\|$ are

\noindent
\begin{tabular}{@{}p{.5\linewidth}@{}p{.5\linewidth}@{}}
	\begin{equation}\label{eq:innerfft}
	\inner{a}{b} = \frac{1}{\deg(\phi)}\sum_{\phi(\zeta)=0} a(\zeta)\cdot \overline{b(\zeta)}
	\end{equation}
	&
	\begin{equation}\label{eq:norm}
	\|a\| = \sqrt{\inner{a}{a}}
	\end{equation}
\end{tabular}

We extend these definitions to vectors: for $\vecu = (u_i)_i$ and $\vecv = (v_i)_i$ in $\cQ^{m}$, $\inner{\vecu}{\vecv} = \sum_i \inner{u_i}{v_i}$.
For our choice of $\phi$, the inner product coincides with the usual coefficient-wise inner product:
\begin{equation}\label{eq:innercoef}
\inner{a}{b} = \sum_{0 \leq i < n} a_i b_i;
\end{equation}
From an algorithmic point of view, computing the inner product or the norm is most easily done by using \eqref{eq:innerfft} if polynomials are in FFT representation, and by using \eqref{eq:innercoef} if they are in coefficient representation.

\paragraph{Ring Lattices.} For the rings $\cQ = \bQ[x]/(\phi)$ and $\cZ = \bZ[x]/(\phi)$, positive integers $m \geq n$ and a full-rank matrix $\matB \in \cQ^{n\times m}$, we denote by $\Lambda(\matB)$ and call lattice generated by $\matB$ the set $\cZ^n \cdot \matB = \{ \vecz \matB | \vecz \in \cZ^{n}\}$. By extension, a set $\Lambda$ is a lattice if there exists a matrix $\matB$ such that $\Lambda = \Lambda(\matB)$. We may say that $\Lambda \subseteq \cZ^m$ is a $q$-ary lattice if $ q\cZ^m \subseteq \Lambda$.

\paragraph{Discrete Gaussians.} For $\sigma, \mu\in \bR$ with $\sigma >0$, we define the Gaussian function $\rho_{\sigma,\mu}$ as $\rho_{\sigma,\mu}(x) = \exp(-|x-\mu|^2/2\sigma^2)$, and the discrete Gaussian distribution $D_{\bZ,\sigma,\mu}$ over the integers as
\begin{equation}
D_{\bZ,\sigma,\mu}(x) = \frac{\rho_{\sigma,\mu}(x)}{ \sum_{z \in \bZ} \rho_{\sigma,\mu}(z) } .
\end{equation}
The parameter $\mu$ may be omitted when it is equal to zero.

%\paragraph{Field norm.} Let $\bK$ be a number field of degree $n = [\bK : \bQ]$ over $\bQ$ and $\bL$ be a Galois extension of $\bK$. We denote by $\gal(\bL/\bK)$ the Galois group of $\bL/\bK$.
%The field norm $\N_{\bL/\bK} : \bL \rightarrow \bK$ is a map defined for any $f \in \bL$ by the product of the Galois conjugates of $f$:
%\begin{equation}
% \N_{\bL/\bK} (f) = \prod_{\g \in \gal(\bL/\bK)} \g(f).
%\end{equation}
%Equivalently, $\N_{\bL/\bK} (f)$ can be defined as the determinant of the $\bK$-linear map $y \in \bL \mapsto f y$. One can check that the field norm is a multiplicative morphism.

%\tprcomment{I commented the paragraph relative to the field norm}

\paragraph{The Gram-Schmidt orthogonalization.}
Any matrix $\matB \in \cQ^{n \times m}$ can be decomposed as follows:
\begin{equation}
 \matB = \matL \times \tBB,
\end{equation}
where $\matL$ is lower triangular with $1$'s on the diagonal, and the rows $\tilde \vecb_i$'s of $\tBB$ verify $\inner{\vecb_i}{ \vecb_j} = 0$ for $i \neq j$. When $\matB$ is full-rank , this decomposition is unique, and it is called the Gram-Schmidt orthogonalization (or \gso).
We will also call Gram-Schmidt norm of $\matB$ the following value:
\begin{equation}
 \gsnorm{\matB} = \max_{\tilde\vecb_i \in \tBB} \|\tilde\vecb_i\|.
\end{equation}

\paragraph{The $\LDLs$ decomposition.} The $\LDLs$ decomposition writes any full-rank Gram matrix as a product $\L \matD \adj\L$, where $\matL \in \cQ^{n\times n}$ is lower triangular with $1$'s on the diagonal, and $\matD \in \cQ^{n\times n}$ is diagonal.

The $\LDLs$ decomposition and the \gso are closely related as for a basis $\matB$, there exists a unique \gso $\matB = \L \cdot \tBB$ and for a full-rank Gram matrix $\matG$, there exists a unique $\LDLs$ decomposition $\matG = \L  \matD  \adj\L$. If $\matG = \matB \adj \matB$, then $\matG = \L \cdot (\tBB \adj \tBB) \cdot \adj\L$ is a valid $\LDLs$ decomposition of $\matG$. As both decompositions are unique, the matrices $\L$ in both cases are actually the same. In a nutshell:
\begin{equation}
 \left[\L\cdot\tBB \text{ is the \gso of } \matB \right]
  \Leftrightarrow  \left[ \L \cdot (\tBB\adj \tBB) \cdot \adj\L  \text{ is the $\LDLs$ decomposition of }(\matB\adj \matB)\right].
\end{equation}
The reason why we present both equivalent decompositions is because the \gso is a more familiar concept in lattice-based cryptography, whereas the use of $\LDLs$ decomposition is faster and therefore makes more sense from an algorithmic point of view.
