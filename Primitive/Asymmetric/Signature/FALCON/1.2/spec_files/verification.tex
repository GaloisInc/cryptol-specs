% @author Thomas Prest
% @author Pierre-Alain Fouque
% @author Jeffrey Hoffstein
% @author Paul Kirchner
% @author Vadim Lyubashevsky
% @author Thomas Pornin
% @author Thomas Ricosset
% @author Gregor Seiler
% @author William Whyte
% @author Zhenfei Zhang
% @copyright The authors

\newpage
\section{Signature Verification} \label{sec:spec:verify}

\subsection{Overview}

The signature verification procedure is much simpler than the key pair generation and the signature generation.
Given a public key $\pk = h$, a message \msg, a signature \signature = (\salt,\comps) and an acceptance bound $\sqsignorm$, the verifier uses \pk to verify that \signature is a valid signature for the message \msg as specified hereinafter:
\begin{enumerate}
 \item
 The value \salt (called ``the salt'') and the message \msg are concatenated to a string $(\salt\|\msg)$ which is hashed to a polynomial $c\in \bZ_q[x]/(\phi)$ as specified by \longhashtopoint.
 \item
 \comps is decoded (decompressed) to a polynomial $s_2 \in \bZ[x]/(\phi)$, see \longdecompress.
 \item
 The value $s_1  = c - s_2 h \bmod q$ is computed.
 \item
 If $\|(s_1,s_2)\|^2 \leq \sqsignorm$, then the signature is accepted as valid. Otherwise, it is rejected.
\end{enumerate}

%We recall that the norm $\|\cdot\|$ is easily computed by using \eqref{eq:norm} with \eqref{eq:innercoef}.

%The only subtlety here is that, as recalled in the notations, $\|\|$
%denotes the embedding norm and not the coefficient norm. However, it is
%possible to compute it in linear time. Given two polynomials $a$ and $b$
%in $\bZ_q[x]/(\phi)$, whose coefficients are denoted $a_j$ and $b_j$,
%respectively, the norm $\|(a,b)\|$ is such that:
%\begin{equation}
%  \|(a,b)\|^2 = \sum_{j=0}^{n-1} (a_j^2 + b_j^2)
%\end{equation}

\subsection{Specification}

The specification of the signature verification is given in \longverify.

\begin{algorithm}%[H]
\caption{\verify(\msg, \signature, \pk, $\sqsignorm$)}\label{alg:verify}
 \begin{algorithmic}[1]
 \Require {A message \msg, a signature $\signature = (\salt, \comps)$, a public key $\pk = h \in \bZ_q[x]/(\phi)$, a bound $\sqsignorm$}
 \Ensure {Accept or reject}
 \State{$c \gets \hashtopoint(\salt\|\msg, q, n)$}
 \State{$s_2 \gets \decompress(\comps, 8 \cdot \sigbytelen - 328)$}
 \If{$(s_2 = \bot)$}\label{line:bots2}
 \Reject  \Comment{Reject invalid encodings}\label{line:rejs2}
 \EndIf
 \State{$s_1 \gets c - s_2 h \bmod q$} \Comment{$s_1$ should be normalized between $\left\lceil - \frac{q}{2} \right\rceil$ and $\left\lfloor \frac{q}{2} \right\rfloor$}
 \If{$\|(s_1,s_2)\|^2 \leq \sqsignorm$}\label{line:sqsignorm}
 \Accept
 \Else
 \Reject \Comment{Reject signatures that are too long}
 \EndIf
 \end{algorithmic}
\end{algorithm}

Computation of $s_1$ can be performed entirely in $\bZ_q[x]/(\phi)$; the
resulting values should then be normalized to the $\lceil -q/2 \rceil$
to $\lfloor q/2 \rfloor$ range.
% /!\ Norm already defined in the notations.
In order to avoid computing a square root, the squared norm can be computed, using only integer operations,
and then compared to $\sqsignorm$.
