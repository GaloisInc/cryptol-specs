% !TeX root = ../falcon.tex

\clearpage

\section{Signature Generation}\label{sec:spec:sign}

\subsection{Overview}

\begin{figure}[!htb]
	\centering
	\begin{tikzpicture}[every node/.style={draw=black}]
	\matrix (m) [matrix of nodes, row sep=7mm, column sep = 1.5cm, draw=none]
	{
		& \sign & \\
		\hashtopoint & \ffsampling & \compress \\
		\shake & \samplerz & \\
		\basesampler & \berexp & \\
		& \approxexp & \\
	};
	\draw[line] (m-1-2) -> (m-2-1);
	\draw[line] (m-1-2) -> (m-2-2);
	\draw[line] (m-1-2) -> (m-2-3);
	\draw[line] (m-2-1) -> (m-3-1);
	\draw[line] (m-2-2) -> (m-3-2);
	\draw[line] (m-3-2) -> (m-4-1);
	\draw[line] (m-3-2) -> (m-4-2);
	\draw[line] (m-4-2) -> (m-5-2);
	\end{tikzpicture}
	\caption{Flowchart of the signature}\label{fig:signature}
\end{figure}

At a high level, the principle of the signature generation algorithm is simple: it first computes a hash value $c \in \bZ_q[x]/(\phi)$ from the message \msg and a salt \salt, and it then uses its knowledge of the secret key $f,g,F,G$ to compute two short values $s_1, s_2$ such that $s_1 + s_2 h = c \bmod q$.

A naive way to find such short values $(s_1, s_2)$ would be to compute $\vect \gets (c,0) \cdot \matB^{-1}$, round it coefficient-wise to a vector $\vecz = \lfloor \vect \rceil$ and output $(s_1, s_2) \gets (\vect - \vecz) \matB$; it fulfils all the requirements to be a legitimate signature, but this method is known to be insecure and to leak the private key.

The proper way to generate $(s_1, s_2)$ without leaking the private key is to use a trapdoor sampler (see \cref{sec:ratio:ffs} for a brief reminder on trapdoor samplers). In \falcon, we use a trapdoor sampler called fast Fourier sampling. The computation of the falcon tree \tree by \longffldl during the key pair generation is the initialization step of this trapdoor sampler.

The heart of our signature generation, \longffsampling applies a randomized rounding (according to a discrete Gaussian distribution) on the coefficients of $\vect$. But it does so in an adaptive manner, using the information stored in the \falcon tree \tree.

At a high level, our fast Fourier sampling algorithm can be seen as a recursive variant of Klein's well known trapdoor sampler (also known as the GPV sampler). Klein's sampler uses a matrix $\matL$ (and the norm of Gram-Schmidt vectors) as a trapdoor, whereas ours uses a tree of such matrices (or rather, a tree of their non-trivial elements).
Given $\vect = (t_0, t_1) \in \bQ[x]/(\phi))^2$, our algorithm first splits $t_1$ using the splitting operator, recursively applies itself to it (using the right child \tree.\rchild of \tree), and uses the merging operator to lift the solution to the ring $\bZ[x]/(\phi)$; it then applies itself again recursively with $t_0$. Note that the recursions cannot be done in parallel: the second recursion takes into account the result of the first recursion, and this is done using information contained in \tree.\data.

The most delicate part of our signature algorithm is the fast Fourier sampling described in \ffsampling, because it makes use of the \falcon tree and of discrete Gaussians over $\bZ$. The rest of the algorithm, including the compression of the signature, is rather straightforward to implement.

Formally, given a private key \sk and a message \msg, the signer uses \sk to sign \msg as follows:
\begin{enumerate}
 \item A random salt \salt is generated uniformly in $\{0, 1\}^{320}$. The concatenated string $(\salt\|\msg)$ is then hashed to a point $c \in \bZ_q[x]/(\phi)$ as specified by \longhashtopoint.
 \item A (not necessarily short) preimage $\vect$ of $c$ is computed, and is then given as input to the fast Fourier sampling algorithm, which outputs two short polynomials $s_1, s_2 \in \bZ[x]/(\phi)$ (in \fft representation) such that $s_1 + s_2 h = c \bmod q$, as specified by \longffsampling.
 \item $s_2$ is encoded (compressed) to a bitstring $\comps$ as specified in \longcompress.
 \item
 The signature consists of the pair $(\salt, \comps)$.
\end{enumerate}

%\tprcomment{I added new algorithms to the flowchart}

%\tprcomment{I added a check on the length/validity of the compressed signature}

\begin{algorithm}[!htb]
	\caption{\sign(\msg, \sk, $\sqsignorm$)}\label{alg:sign}
	\begin{algorithmic}[1]
	\Require {A message \msg, a secret key \sk, a bound $\sqsignorm$}
	\Ensure {A signature \signature of \msg}
	\State{$\salt \gets \{0, 1\}^{320}$ uniformly}
	\State{$c \gets \hashtopoint(\salt\|\msg, q, n)$}
	\State{$\vect \gets \left(- \frac{1}{q} \fft(c) \odot \fft(F), \frac{1}{q} \fft(c) \odot \fft(f) \right)$}\label{line:t}
	\Comment{$\vect = (\fft(c), \fft(0)) \cdot \hat\matB^{-1}$}
	\Do\label{line:do}
	\Do
	\State{$\vecz \gets \ffsampling_n(\vect, \tree)$}
	\State{$\vecs = (\vect - \vecz)  \hat\matB$}
	\Comment{At this point, $\vecs$ follows a Gaussian distribution: $\vecs \sim D_{(c, 0) + \Lambda(\matB), \sigma, 0}$}
	\doWhile{$\|\vecs\|^2 > \sqsignorm$}\label{line:sqsig}
	\Comment{Since $\vecs$ is in \fft representation, one may use \eqref{eq:innerfft} to compute $\|\vecs\|^2$}
	\State{$(s_1, s_2) \gets \ifft(\vecs)$}\Comment{$s_1 + s_2 h = c \bmod (\phi, q)$}
	\State{$\textsf{s} \gets \compress(s_2, 8 \cdot \sigbytelen - 328)$}
	\Comment{Remove $1$ byte for the header, and $40$ bytes for \salt}
	\doWhile{$(\textsf{s} = \bot)$}\label{line:while}
	\Return{$\signature = (\salt, \textsf{s})$}
	\end{algorithmic}
\end{algorithm}


%\tprcomment{Commented some text}

%\paragraph{A note on sampling over $\bZ$.} Algorithm~\ref{alg:ffsampling} requires access to an oracle $\cD$ for the distribution $D_{\bZ, \sigma', c'}$, where $\sigma'$ can be the value of any leaf of the private \falcon tree \tree, and $c' \in \bQ$ is arbitrary. How to implement $\cD$ is outside the scope of this specification. It is only required that the R\'enyi divergence between this oracle and an ideal discrete Gaussian $D_{\bZ, \sigma', c'}$ verifies $R_{512}(\cD \| D_{\bZ, \sigma', c'}) \leq 1 + 2^{-66}$, for the definition of the R\'enyi divergence given in \eg \cite{AC:BLLSS15}. We note that several proposals~\cite{EPRINT:ZWXZ18,EPRINT:ZhaSteSak18,DAC:KSVV19,EPRINT:Walter19} for efficient (constant-time) Gaussian sampling over the integers have been made recently.
%
%Our reference implementation uses a Gaussian sampler based on
%rejection sampling against a bimodal distribution; it is described in
%Section~\ref{sec:impl:gaussian}. We note that the range of
%possible values for the standard deviation in the Gaussian sampler is
%limited: it is always greater than $1.2$, and always lower than $1.9$.



 \subsection{Fast Fourier Sampling}

 This section describes our fast Fourier sampler: \longffsampling. We note that we perform all the operations in \fft representation for efficiency reasons, but the whole algorithm could also be executed in coefficient representation instead, at a price of a $O(\log n)$ penalty in speed.

% \tprcomment{I inserted \samplerz in \ffsampling}

 \begin{algorithm}[!htb]
  \caption{$\ffsampling_{\ n}(\vect, \tree)$}\label{alg:ffsampling}
 \begin{algorithmic}[1]
  \Require {$\vect=(t_0, t_1) \in \fft\left(\bQ[x]/(x^n+1)\right)^2$, a \falcon tree $\tree$}
  \Ensure {$\vecz=(z_0, z_1) \in \fft\left(\bZ[x]/(x^n+1)\right)^2$}% such that $\vecz \cdot \matB \sim D_{\Lambda(\matB), \sigma, \vect \cdot \matB}$ for a given $\sigma$}
  \Format{All polynomials are in \fft representation.}
  \If{$n=1$}
  \State{$\sigma' \gets \tree.\data$}
  \Comment{It is always the case that $\sigma' \in [\sigmin, \sigmax]$}
  \State{$ z_0 \gets \samplerz(t_0, \sigma')$}\label{line:samplerz0}
  \Comment{Since $n=1$, $ t_i = \ifft( t_i) \in \bQ$ and $ z_i = \ifft( z_i) \in \bZ$}
  \State{$ z_1 \gets \samplerz(t_1, \sigma')$}\label{line:samplerz1}
%  \Comment{Since $n=1$, $ t_1 = \ifft( t_1) \in \bQ$ and $ z_1 = \ifft( z_1) \in \bZ$}
  \Return{$ \vecz = (z_0, z_1)$}
  \EndIf
  \State{$(\ell, \tree_0, \tree_1) \gets (\tree.\data, \tree.\lchild, \tree.\rchild)$}
  \State{$\vect_1 \gets \splitfft(t_1)$}\Comment{$\vect_0, \vect_1 \in \fft\left(\bQ[x]/(x^{n/2}+1)\right)^2$}
  \State{$ \vecz_1 \gets \ffsampling_{\ n/2}(\vect_1, \tree_1)$}\Comment{First recursive call to $\ffsampling_{\ n/2}$}
  \State{$ z_1 \gets \mergefft(\vecz_1)$}\Comment{$\vecz_0, \vecz_1 \in \fft\left(\bZ[x]/(x^{n/2}+1)\right)^2$}
  \State{$ t_0' \gets  t_0 + ( t_1 -  z_1) \fdot  \ell$}
  \State{$\vect_0 \gets \splitfft(t_0')$}
  \State{$ \vecz_0 \gets \ffsampling_{\ n/2}(\vect_0, \tree_0)$}\Comment{Second recursive call to $\ffsampling_{\ n/2}$}
  \State{$z_0 \gets \mergefft(\vecz_0)$}
  \Return{$ \vecz = ( z_0, z_1)$}
  \end{algorithmic}
 \end{algorithm}

\newpage

\subsection{Sampler over the Integers}\label{sec:spec:sign:integers}

\newcommand{\rightshift}{\ \texttt{>>}\ }

Let $1 \leq \sigmin < \sigmax$. This section shows how to sample securely Gaussian samples $z \sim D_{\bZ, \sigma', \mu}$ for any $\sigma' \in [\sigmin, \sigmax]$ and $\mu \in \cR$. This is done by \longsamplerz, which calls \longbasesampler and \longberexp as subroutines. We use the notations (\rightshift) and (\texttt{\&}) to denote the bitwise right-shift and AND, respectively. We also introduce the notations $\istrue{\cdot}$ and $\uniform$:

\begin{equation}
\text{For any logical proposition }P, \quad \istrue{P} =
\begin{cases}
1 & \text{if $P$ is true} \\
0 & \text{if $P$ is false} \\
\end{cases}
\end{equation}
Note that $\istrue{\cdot}$ needs to be realized in constant time for our algorithms to be resistant against timing attacks.

\begin{equation}\label{eq:uniform}
\forall k \in \bZ^+, \quad \uniform(k) \text{ samples $z$ uniformly in $\{0, 1, ..., 2^k - 1\}$.}
\end{equation}


\paragraph{\basesampler.} Let \pdt be as in \cref{tab:chi}. Our first procedure is \longbasesampler. It samples an integer $z_0 \in \bZ^+$ according to the distribution $\chi$ of support $\{0, \dots, \cdtlen\}$ uniquely defined as:
\begin{equation}\label{eq:chi}
\forall i \in \{0, \dots, \cdtlen\}, \qquad \chi(i) = 2^{-72} \cdot \pdt[i]
\end{equation}
%One can check that $\sum_{i=0}^{\cdtlen} \pdt[i] = 2^{72}$, so that \eqref{eq:chi} does define a unique distribution.
The distribution $\chi$ is extremely close to the ``half-Gaussian'' $D_{\bZ^+, \sigmax}$ in the sense that $R_{513}(\chi \| D_{\bZ^+, \sigmax}) \leq 1 + 2^{-78}$, where $R_*$ is the R\'enyi divergence. For completeness, \cref{tab:chi} provides the values of:
\begin{itemize}[nolistsep,noitemsep]
	\item the (scaled) probability distribution table $\pdt[i]$;
	\item the (scaled) cumulative distribution table $\cdt[i] = \sum_{j\leq i} \pdt[j]$;
	\item the (scaled) reverse cumulative distribution table $\rcdt[i] = \sum_{j > i} \pdt[j] = 2^{72} - \cdt[i]$.
\end{itemize}

{%
\nprounddigits{0}
\begin{table}[!htb]
\centering
\caption{Values of the \{probability/cumulative/reverse cumulative\} distribution table for the distribution $\chi$, scaled by a factor $2^{72}$.}\label{tab:chi}
\medskip
{\small
\begin{tabular}{l|>{\ttfamily}r|>{\ttfamily}r|>{\ttfamily}r}
\texttt{i} & \pdt{}[i] & \cdt{}[i] & \rcdt{}[i] \\
\hline
0 & \numprint{1697680241746640300030} & \numprint{1697680241746640300030} & \numprint{3024686241123004913666} \\
1 & \numprint{1459943456642912959616} & \numprint{3157623698389553259646} & \numprint{1564742784480091954050} \\
2 & \numprint{928488355018011056515} & \numprint{4086112053407564316161} & \numprint{636254429462080897535} \\
3 & \numprint{436693944817054414619} & \numprint{4522805998224618730780} & \numprint{199560484645026482916} \\
4 & \numprint{151893140790369201013} & \numprint{4674699139014987931793} & \numprint{47667343854657281903} \\
5 & \numprint{39071441848292237840} & \numprint{4713770580863280169633} & \numprint{8595902006365044063} \\
6 & \numprint{7432604049020375675} & \numprint{4721203184912300545308} & \numprint{1163297957344668388} \\
7 & \numprint{1045641569992574730} & \numprint{4722248826482293120038} & \numprint{117656387352093658} \\
8 & \numprint{108788995549429682} & \numprint{4722357615477842549720} & \numprint{8867391802663976} \\
9 & \numprint{8370422445201343} & \numprint{4722365985900287751063} & \numprint{496969357462633} \\
10 & \numprint{476288472308334} & \numprint{4722366462188760059397} & \numprint{20680885154299} \\
11 & \numprint{20042553305308} & \numprint{4722366482231313364705} & \numprint{638331848991} \\
12 & \numprint{623729532807} & \numprint{4722366482855042897512} & \numprint{14602316184} \\
13 & \numprint{14354889437} & \numprint{4722366482869397786949} & \numprint{247426747} \\
14 & \numprint{244322621} & \numprint{4722366482869642109570} & \numprint{3104126} \\
15 & \numprint{3075302} & \numprint{4722366482869645184872} & \numprint{28824} \\
16 & \numprint{28626} & \numprint{4722366482869645213498} & \numprint{198} \\
17 & \numprint{197} & \numprint{4722366482869645213695} & \numprint{1} \\
18 & \numprint{1} & \numprint{4722366482869645213696} & \numprint{0} \\
\end{tabular}}
\end{table}
}

\begin{algorithm}[!htb]
	\caption{$\basesampler()$}\label{alg:basesampler}
	\begin{algorithmic}[1]
		\Require {-}
		\Ensure {An integer $z_0 \in \{0, \dots, \cdtlen\}$ such that $z \sim \chi$} \Comment{$\chi$ is uniquely defined by \eqref{eq:chi}}
		\State{$u \gets \uniform(72)$}\label{line:basesampler} \Comment{See \eqref{eq:uniform}}
		\State{$z_0 \gets 0$}
		\For{$i = 0, \dots, \cdtlenminus$}
		\State{$z_0 \gets z_0 + \llbracket u < \rcdt[i] \rrbracket$} \Comment{Note that one should use \rcdt, not \pdt or \cdt}
		\EndFor
		\Return{$z_0$}
	\end{algorithmic}
\end{algorithm}



\paragraph{\berexp and \approxexp.} \longberexp and its subroutine \longapproxexp serve to perform rejection sampling. Let $C$ be the following list of 64-bit numbers (in hexadecimal form):

\begin{align*}
C = & \footnotesize{\texttt{%
	[0x00000004741183A3, 0x00000036548CFC06, 0x0000024FDCBF140A, 0x0000171D939DE045,}} \\
	& \footnotesize{\texttt{%
	\ 0x0000D00CF58F6F84, 0x000680681CF796E3, 0x002D82D8305B0FEA, 0x011111110E066FD0,}}\\%
	& \footnotesize{\texttt{%
	\ 0x0555555555070F00, 0x155555555581FF00, 0x400000000002B400, 0x7FFFFFFFFFFF4800,}}\\%
	& \footnotesize{\texttt{%
	\ 0x8000000000000000]}}.%
\end{align*}
Let $f \in \bR[x]$ be the polynomial defined as:
$$f(x) = 2^{-63} \cdot \sum_{i=0}^{\poldeg} C[i] \cdot x^{\poldeg - i}.$$
$f(-x)$ serves as a very good approximation of $\exp(-x)$ over $[0, \ln(2)]$, see~\cite{TC:ZhaSteSak20}. This is leveraged by \longapproxexp to compute integral approximations of $2^{63} \cdot ccs \cdot \exp(-x)$ for $x$ in a certain range. Note that the intermediate variables $y, z$ in \approxexp are always in the range $\{0, ..., 2^{63} - 1\}$, with one exception: if $x = 0$, then at the end of the for loop (\cref{step:approxloop,step:approxloopin}) we have $y = 2^{63}$. This makes it easy to represent $x,y$ using, for example, the C type \verb+uint64_t+.

\todo[inline]{Fat warning: check that \approxexp is correct}

\begin{algorithm}[!htb]
	\caption{$\approxexp(x, ccs)$}\label{alg:approxexp}
	\begin{algorithmic}[1]
		\Require {Floating-point values $x \in [0, \ln(2)]$ and $ccs \in [0, 1]$}
		\Ensure {An integral approximation of $2^{63} \cdot ccs \cdot \exp(-x)$}
		\State{$C$ = {\footnotesize\texttt{%
		[0x00000004741183A3,0x00000036548CFC06,0x0000024FDCBF140A,0x0000171D939DE045, 0x0000D00CF58F6F84, 0x000680681CF796E3, 0x002D82D8305B0FEA, 0x011111110E066FD0, 0x0555555555070F00, 0x155555555581FF00, 0x400000000002B400, 0x7FFFFFFFFFFF4800,	0x8000000000000000]%
		}}}
%	 \Comment{$C$ is indexed from $0$ to $\poldeg$}
		\State{$y \gets C[0]$}\Comment{$y$ and $z$ remain in $\{0, ..., 2^{63} - 1\}$ the whole algorithm.}
		\State{$z \gets \lfloor 2^{63} \cdot x \rfloor$}
		 %\Comment{Is that what the code is doing??}
		\For{$1 = 1, \dots, \poldeg$}\label{step:approxloop}
		\State{$y \gets C[u] -  (z \cdot y) \rightshift 63$}\label{step:approxloopin}
		\Comment{$(z \cdot y)$ fits in 126 bits, but we only need the top 63 bits}
		\EndFor
		\State{$z \gets \lfloor 2^{63} \cdot ccs \rfloor$}
		\State{$y \gets (z \cdot y) \rightshift 63$}
%		\Comment{Does it differ from the C code (\rightshift 64)??}
		\Return{$y$}
	\end{algorithmic}
\end{algorithm}

Given inputs $x, ccs \geq 0$, \longberexp returns a single bit 1 with probability $\approx ccs \cdot \exp(-x)$.

\begin{algorithm}[!htb]
	\caption{$\berexp(x, ccs)$}\label{alg:berexp}
	\begin{algorithmic}[1]
		\Require {Floating point values $x, ccs \geq 0$}
		\Ensure {A single bit, equal to 1 with probability $\approx ccs \cdot \exp(-x)$}
		\State{$s \gets \lfloor x / \ln(2) \rfloor$} \Comment{Compute the unique decomposition $x = 2^s \cdot r$, with $(r, s) \in [0, \ln 2) \times \bZ^+$}
		\State{$r \gets x - s \cdot \ln(2)$}
		\State{$s \gets \min(s, 63)$}
		\State{$z \gets (2 \cdot \approxexp(r, ccs) - 1) \rightshift s$}
		\Comment{$z \approx 2^{64-s} \cdot ccs \cdot \exp(-r) = 2^{64} \cdot ccs \cdot \exp(-x)$}
		\State{$i \gets 64$}
		\Do
		\State{$i \gets i - 8$}
		\State{$w \gets \uniform(8) - \left((z \rightshift  i)\ \texttt{\&} \ \texttt{0xFF}\right)$} \label{line:berexp}
		\doWhile{($(w = 0)$ and $(i > 0)$)} \Comment{This loop does not need to be done in constant-time}
		\Return{$\llbracket w < 0 \rrbracket $} \Comment{Return 1 with probability $2^{-64} \cdot z \approx ccs \cdot \exp(-x)$}
	\end{algorithmic}
\end{algorithm}

\paragraph{\samplerz.} Finally, \longsamplerz use the previous algorithms as subroutines and, given inputs $\mu, \sigma'$ in a certain range, outputs an integer $z \sim D_{\bZ, \sigma', \mu}$ in an isochronous manner.

\begin{algorithm}[!htb]
	\caption{$\samplerz(\mu, \sigma')$}\label{alg:samplerz}
	\begin{algorithmic}[1]
		\Require {Floating-point values $\mu, \sigma' \in \cR$ such that $\sigma' \in [\sigmin, \sigmax]$}
		\Ensure {An integer $z \in \bZ$ sampled from a distribution very close to $D_{\bZ, \mu, \sigma'}$}
		\State{$r \gets \mu - \lfloor \mu \rfloor$}
		\Comment{$r$ must be in $[0, 1)$}
		% \State{$dss \gets 1/(2\sigma'^2)$}
		\State{$ccs \gets \sigmin / \sigma'$} \Comment{$ccs$ helps to make the algorithm running time independent of $\sigma'$}
		\While{(1)}
		\State{$z_0 \gets \basesampler()$}
		\State{$b \gets \uniform(8)\ \texttt{\&}\ \texttt{0x1}$}\label{line:sign}
		\State{$z \gets b + (2 \cdot b - 1) z_0$}
		\State{$x \gets \frac{(z - r)^2}{2 \sigma'^2} - \frac{z_0^2}{2 \sigmax^2}$}
		\If{$(\berexp(x, ccs) = 1)$}
		\Return{$z + \lfloor \mu \rfloor$}
		\EndIf
		\EndWhile
	\end{algorithmic}
\end{algorithm}

%\tprcomment{TODO: add test vectors}

% !TeX root = ../falcon.tex

\newcommand{\randombytes}{\texttt{randombytes}\xspace}

\paragraph{Known Answer Tests (KAT).} To help the proper implementation of \longsamplerz and its subroutines, \cref{tab:kat} provides test vectors. Let $\sigmin = \sigminvali$ (the value of $\sigmin$ for \falcon-512). Each line of \cref{tab:kat} provides a tuple $(\mu, \sigma', \randombytes, z)$ such that when replacing internal calls to $\uniform()$ with reading bytes from \randombytes (acting as a random bytestring):
\begin{equation}\label{eq:samplerz}
	 \samplerz(\mu, \sigma') \rightarrow z
\end{equation}

For readability, \cref{tab:kat} splits \randombytes according to each iteration of \samplerz. As an example, line 1 of \cref{tab:kat} indicates that for $\mu = -91.90471153063714$, $\sigma' = 1.7037990414754918$, \randombytes = {\small \texttt{0fc5442ff043d66e91d1eacac64ea5450a22941edc6c}} and $z = -92$ , the equation \eqref{eq:samplerz} is verified when running \samplerz with randomness \randombytes. In addition, \samplerz iterates twice before terminating. More precisely, \randombytes is used as follows:
\[
\underbrace{\tt 0fc5442ff043d66e91|d1|ea}_{\text{Iteration 1}}|
\underbrace{\tt cac64ea5450a22941e|dc|6c}_{\text{Iteration 2}}
\]
At each iteration, the first $9$ random bytes are used by \basesampler, the next one by \cref{line:sign} and the last one(s) by \berexp. Note that at each call, \berexp has a probability $\frac{1}{2^8}$ of using more than $1$ random byte; this is rare, but happens. This is illustrated by line 9 of \cref{tab:kat}, which contain an example for which one iteration of \berexp uses $2$ random bytes.

For further testing, this submission package contains more extensive and detailed test vectors. See:
\vspace{-4mm}
\begin{center}
{\small\tt Supporting\_Documentation/additional/test-vector-sampler-falcon\{512,1024\}.txt}
\end{center}


\begin{table}[htb!]
\caption{Test vectors for \samplerz ($\sigmin = \sigminvali$)}\label{tab:kat}

\bigskip

\begin{tabular} {@{\makebox[1.5em][l]{\rownumber\space}}|>{\ttfamily}l | >{\ttfamily}l |>{\ttfamily}p{51mm}|>{\ttfamily}r}
	\gdef\rownumber{\stepcounter{magicrownumbers}\arabic{magicrownumbers}} {\normalfont Center $\mu$} & {\normalfont Standard deviation $\sigma'$} & {\normalfont \randombytes} & {\normalfont Output $z$}\\
\hline
\hline
-91.90471153063714      & 1.7037990414754918    & 0fc5442ff043d66e91d1ea cac64ea5450a22941edc6c  & -92 \\
\hline
-8.322564895434937      & 1.7037990414754918    & f4da0f8d8444d1a77265c2 ef6f98bbbb4bee7db8d9b3  & -8 \\
\hline
-19.096516109216804     & 1.7035823083824078    & db47f6d7fb9b19f25c36d6 b9334d477a8bc0be68145d  & -20 \\
\hline
-11.335543982423326     & 1.7035823083824078    & ae41b4f5209665c74d00dc c1a8168a7bb516b3190cb4 2c1ded26cd52aed770eca7 dd334e0547bcc3c163ce0b  & -12 \\
\hline
7.9386734193997555      & 1.6984647769450156    & 31054166c1012780c603ae 9b833cec73f2f41ca5807c c89c92158834632f9b1555  & 8 \\
\hline
-28.990850086867255     & 1.6984647769450156    & 737e9d68a50a06dbbc6477  & -30 \\
\hline
-9.071257914091655      & 1.6980782114808988    & a98ddd14bf0bf22061d632  & -10 \\
\hline
-43.88754568839566      & 1.6980782114808988    & 3cbf6818a68f7ab9991514  & -41 \\
\hline
-58.17435547946095      & 1.7010983419195522    & 6f8633f5bfa5d26848668e 3d5ddd46958e97630410587c  & -61 \\
\hline
-43.58664906684732      & 1.7010983419195522    & 272bc6c25f5c5ee53f83c4 3a361fbc7cc91dc783e20a  & -46 \\
\hline
-34.70565203313315      & 1.7009387219711465    & 45443c59574c2c3b07e2e1 d9071e6d133dbe32754b0a  & -34 \\
\hline
-44.36009577368896      & 1.7009387219711465    & 6ac116ed60c258e2cbaeab 728c4823e6da36e18d08da 5d0cc104e21cc7fd1f5ca8 d9dbb675266c928448059e  & -44 \\
\hline
-21.783037079346236     & 1.6958406126012802    & 68163bc1e2cbf3e18e7426  & -23 \\
\hline
-39.68827784633828      & 1.6958406126012802    & d6a1b51d76222a705a0259  & -40 \\
\hline
-18.488607061056847     & 1.6955259305261838    & f0523bfaa8a394bf4ea5c1 0f842366fde286d6a30803  & -22 \\
\hline
-48.39610939101591      & 1.6955259305261838    & 87bd87e63374cee62127fc 6931104aab64f136a0485b  & -50 \\
\hline
%-101.00385632924957     & 1.694194291360674     & e72eadbc08ea77ed1c2823  & -102 \\
%\hline
%-55.766206519422155     & 1.694194291360674     & 3632c29bef5ff255bbba7d  & -58 \\
%\hline
%4.279820924390407       & 1.6937038942280862    & 11e8fbad926a8748efbd3d  & 8 \\
%\hline
%-6.1632330276740515     & 1.6937038942280862    & 755d7eec0dec4ab547669a 44d5113b6d8465102827bb 68fad1b91b1f32c7d65cf2 b27a2de77f5b02549f7829  & -7 \\
%\hline
%-12.636901236335072     & 1.6893009634650535    & 01b2bd367ee80fccf335ae 8ffdf86c0ef4ad076d7854  & -14 \\
%\hline
%-14.19051757550942      & 1.6893009634650535    & 99042f67f18f2a49baeea6 cdba65ef008be154fd9dfd ee32c97f885d20eefe4100  & -14 \\
%\hline
%15.996015013814931      & 1.6885857180002772    & f05c53d4ad1bcf824a4abb 701814bd9cb8b371715ace 3acfdc88a5af541f306e00  & 13 \\
%\hline
%-62.516938164862246     & 1.6885857180002772    & d33fea2db82a9a0d81cec7 455358a2e97b4b914ec392 5da04d64bb3ca6d69ec4f3 87310edf97fa43ceef7490  & -64 \\
%\hline
%-19.134957285144306     & 1.6907818034679203    & d229c9018b2c3c8645bd71  & -19 \\
%\hline
%5.121244097140291       & 1.6907818034679203    & 93f090cdb2164b5584ca95  & 4 \\
%\hline
%6.686630568251369       & 1.6903655284346046    & 97523a0d9b5ae3be81553f  & 8 \\
%\hline
%-37.88347616745064      & 1.6903655284346046    & 2b931fcd0c760fc8515a2a  & -40 \\
%\hline
%-6.163535730572427      & 1.6860138676808625    & 25589b7a9393883dcd8d90  & -4 \\
%\hline
%-12.9593836429217       & 1.6860138676808625    & ba7377dda60733a9416e5b  & -13 \\
%\hline
%5.996166146913          & 1.6853825380694936    & 0188370dcf1fdbaa9c4f35  & 11 \\
\end{tabular}
\end{table}
