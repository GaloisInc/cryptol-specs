% @author Daniel J. Bernstein
% @author Andreas Hülsing
% @author Stefan Kölbl
% @author Ruben Niederhagen
% @author Joost Rijneveld
% @author Peter Schwabe
% @copyright The authors
\section{Design rationale}
% \TODO{Essentially argue that we use the same reasoning as for SPHINCS,
% repeat / sketch that reasoning. Explain changes we made and why we made them.
% Explain ideas we had which got discarded and why. }

The design rationale behind \spx is to follow the original \spc construction
and apply several results from more recent research. The idea behind \spc was
as follows. One can build a
stateless hash-based signature scheme using a massive binary
certification tree and selecting a leaf at random
for each message to be signed. The problem with this approach is that the
tree has to be extremely high, i.e., a height of about twice the security level
would be necessary. This leads to totally unpractical signature sizes. Using a
hypertree instead of a binary certification tree allows to trade speed for
signature size. However, this is still not sufficient to get practical sizes
and speed.

The main new idea in \spc was to not use the leaves directly to
sign messages but to use the leaves to certify FTS key pairs. This allowed to
massively reduce the total tree height (by a factor about $4$). This is due to
the fact that the security of an FTS instance degrades with every signature
a key pair is used for. Hence, the height of the tree does not have to be such
that collisions do only occur with negligible probability anymore. Instead,
it has to be ensured that the product of the probability of a $\gamma$-times
collision on a leaf and the forging probability of an adversary after seeing
$\gamma$ FTS signatures (with the same key pair) is negligible.

From this, it is mainly a question of balancing parameters to find a practical
scheme. For the full original reasoning see~\cite{Bernstein2015}.

In the following we give a more detailed reasoning regarding the changes
made to \spc in \spx, %further decisions made on the way
and changes that were
discussed by the \spx team but got discarded.

\subsection{Changes Made}
We changed several details of \spc leading to \spx. The reasoning behind those
changes is discussed in the following.
\subsubsection{Multi-Target Attack Protection}
\spc was designed to be collision-resilient i.e., to not be vulnerable to
collision attacks against the used hash function. This had two reasons. First,
it allowed to choose a smaller output length at the same security level which
led to smaller signatures. Second, collision resistance is a far stronger
assumption than the used (second-)preimage resistance and pseudorandomness
assumptions.

However, the use of (second-)preimage resistance introduced a new issue as
pointed out in~\cite{Huelsing2016}: Multi-target attacks. Preimage resistance
properties are targeted properties. An adversary is asked to invert the function
on a given target value, or to find a second-preimage for a given target value.
If it suffices to break the given property for one out of many targets, the
adversarial effort is reduced by a factor of the number of targets. To prevent
this in our we apply the mitigation techniques from \cite{Huelsing2016} using keyed
hash functions. Each hash function call is keyed with a different key and applies
different bitmasks. Keys are derived from, and bitmasks are
pseudorandomly generated from a public seed and an address specifying the
context of the call. For this we introduce the notion of tweakable hash functions
which take in addition to the input value a public seed and an address.

This pseudorandom generation of bitmasks comes at the cost of introducing a
random oracle assumption for the PRF used to generate the bitmasks. However,
this only applies to the pseudorandom generation of the bitmasks. I.e., if
all bitmasks would be stored in the public key, the scheme would have a standard
model security proof (even if these bitmasks where generated using exactly the
same way but without giving away the seed). Hence, the security reduction
in~\cite{Huelsing2016} is in the quantum-accessible random oracle model.

One difference to \cite{Huelsing2016} is that in all instantiations of \spx,
keys are not pseudorandomly generated. Instead, the concatenation of
public seed and address is used to practically key the functions.
Given how the tweakable hash functions are instantiated, this means that we
assume that there do not exist any (exponentially large) subsets of the domain
on which second-preimage finding is easy. This assumption holds for any hash
function based on the sponge or Merkle-D{\aa}mgard construction, assuming the
block or compression function behaves like a random function.

\subsubsection{Tree-less \wotsp Public Key Compression}
\spx compresses the end nodes of the \wotsp hash chains with a single call
to a tweakable hash function, while \spc used a so called L-tree.
The reason to use L-trees in \spc was that this required only two
$n$-byte bitmasks per layer, i.e., $2\ceil{\log \len}$ bitmasks. A single call
to a tweakable hash requires $\len$ $n$-byte bitmasks.
As the bitmasks were stored in the public key, this meant
smaller public keys. Now, that bitmasks are pseudorandomly generated anyway and
hence are not stored in the public key anymore,
this argument does not apply. On the opposite, tree based compression
is slower than using a single call to a tweakable hash with longer input.

\subsubsection{\fors} \label{sec:changes:fors}
\fors was used to replace \horst. \horst, as its predecessor
\hors, had the problem that weak messages existed as recently independently
pointed out in \cite{Aumasson2017}. More specifically, in \horst the message
is also split into $k$ indexes as for \fors. However, these indexes all selected
values from the same single set of secret key values. Hence, if the same index
appeared multiple times in a signature, still only a single secret value would be
required. In extreme cases this means that for the signature of a message only
a single secret value has to be know. \fors prevents this using separate secret
value sets per index obtained from the message. Even if a message maps $k$-times
to the same index, the signature now contains $k$ different secret values.

For the same parameters $k$ and $t$ this would mean an increase in signature
size and worse speed as now $k$ trees of height $\log t$ have to be computed
instead of one and for each signature value an authentication path of length
$(\log t) -1$ is needed. However, due to the strengthened security, we can
choose different values for $k$ and $t$. This in the end leads to smaller
signatures than for \horst.

We also considered a method similar to Octopus~\cite{Aumasson2017a}. The idea
is that authentication paths in \horst largely overlap. Hence, it becomes
possible to reduce the signature size removing any redundancy in the
authentication paths. This comes at the cost of a rather involved method to
collect the right nodes as well as variable size signatures. In practice this
means that one still has to prepare for the worst case. This worst case
indeed still has smaller signatures than \horst. We decided against this option
as the \fors signature size matches that of Octopus' worst case signature size.
At the same time, \fors gives more flexibility in the choice of $k$ and $t$, and
comes with a far simpler signature and verification method that Octopus.

\subsubsection{Verifiable Index Selection}
In \spc the index of the \horst instance to be used was pseudorandomly selected.
This had the drawback that the index appeared random to a verifier and it was
impossible to verify that the index was indeed generated that way. This allowed
an adversary a multi-target attack on \horst (similarly for \fors in \spx). An
adversary could first map a message to an index set and then check if the
necessary secret values were already uncovered for some \horst key pair. Then it
would just select the index of that \horst key pair as index and succeed in
forging a signature.

To prevent this attack, we decided to make index generation verifiable. More
specifically, we generate the index together with the message digest:

 We compute message digest and index as
        $$( \md || \idx ) = \sphincsHmsg ( \Random, \PK, \msg )$$
 where $\PK = ( \pseed || \proot )$ contains the top root node and the public seed.

 This way, an adversary can no longer freely choose an index. Indeed, selecting
 a message immediately also fixes the index. This method has another advantage
 in addition to avoiding the multi-target
 attack against \fors /\horst. We can omit the index in the SPHINCS
 signature as it would be redundant.

 \subsubsection{Making Deterministic Signing Optional}
 \label{subsec:optrand}
 The pseudorandom generation of randomizer $\Random$ now allows to use additional
 randomness. It takes a $n$-byte value $\texttt{OptRand}$. Per default
 $\texttt{OptRand}$ is set to 0 but it can be filled
 with random bits e.g. taken from a TRNG. The randomizer is then computed as

        $$\Random = \sphincsPRF (\skprf, \texttt{OptRand}, \msg).$$

 That way, deterministic signing becomes optional. Deterministic signing can be
 a problem for devices which are susceptible to side-channel attacks
 as it allows to collect several traces
 for the exactly same computation by just asking for a signature on the same
 message multiple times.

 We could of course also have replaced $\Random$ by a truly random value on
 default. This would have caused the scheme to become susceptible to bad
 randomness. The new method prevents this. If $\texttt{OptRand}$ is a high
 entropy string, \Random has as much entropy as that string. If
 $\texttt{OptRand}$ is left as zero or has only little entropy, \Random
 is just a pseudorandom value as in \spc.


% \subsection{Decision Made}
%  \TODO{I am not sure if we want the following in here or rather ``forget about it''.}
%  \TODO{I think it is valuable to explain that the XMSS optimizations (like the tweakable hashes)
%  proposed here to improve the SPHINCS performance are not part of the IETF XMSS draft. I propose
% to rename this subsection "Compatibility with IETF XMSS draft" - Panos }
% We want to highlight one decision made when deciding on the details of
% \spx. A big advantage of \spc-type schemes is the large overlap with stateful
% schemes like XMSS($^{MT}$). Especially the verification of standalone XMSS-style schemes
% is essentially just a combination of \hyper signature verification and the
% initial randomized message hash. Hence, it was an obvious question if \spx
% should be compatible with the current CFRG Internet Draft on
% \xmss~\cite{Huelsing2015}. It turned out that some decisions made in the \xmss
% draft were not optimal in hind sight. We decided to go for the optimal choices.
% One of the main points for this decision was that one of the incompatibility
% comes from the way we abstract and instantiate the tweakable hash functions.
% This has a huge impact on speed. This is far more an issue for \spx than for
% \xmss.

\subsubsection{\spx-'simple' and \spx-'robust'}

The updated, Round 2 submission of \spx introduces instantiations of the tweakable hash functions similar to those of the LMS proposal for stateful hash-based signatures~\cite{LMSdraft}. These instantiations are called 'simple' (compared to the established instantiations which we now call 'robust'). The 'simple' instantiations omit the use of bitmasks, i.e., no bitmasks have to be generated and XORed with the message input
of the tweakable hash functions \sphincsF, \sphincsH or \sphincsT.
%Instead it uses the seed \pseed, address \adrs and message as the input
%to the tweakable hash function.
This has the advantage of better speed since the calls to
the underlying hash function (needed in order to generate the bitmasks for each tweakable hash calculation) are saved. However, the resulting drawback is a security argument which in its entirety only applies in the random oracle model.

Another reason to propose these simple instantiations is the possibility to align the construction
with the stateful scheme \cite{LMSdraft} such that clients can easily implement the verification
procedure for both with a small code-base, as for the robust instantiations and XMSS. However, the simple instantiations of \spx are so far not compatible with the LMS signature scheme as described in~\cite{LMSdraft}. The simple instantiations of \spx uses \pseed and \adrs to distinguish hash calls. LMS uses a specially crafted security string which has the same purpose, is similar, but differs in the details.

Most of the time in \spx, XMSS, and LMS, is spent on \sphincsF computations. The LMS proposal \cite{LMSdraft} optimized the length of their security string for \shatwo to ensure that the \sphincsF computations of the OTS signatures can be done with a single compression function call.
We use a similar approach, applied to our \spx-\shatwo instantiation. For this purpose we compress the hash addresses in case of \shatwo instantiations and pad \pseed to fit a full compression function block (with an exception of the mask generation). As \pseed is constant for a key pair, this allows to precompute the internal state of \shatwo after absorbing this block and reduce the necessary online computations to a single compression function call for the \shatwo-simple instances. Also for the robust instantiations this saves a factor of two in compression function calls. For \shathree and \haraka such an optimization is of no effect as one \sphincsF computation already takes only a single call to the inner function.

\subsection{Discarded Changes}
In \autoref{sec:changes:fors}, we already explained that we discarded the use of
an Octopus-like method as we found a better alternative.

One more idea which we discarded on the way was a
signature - secret key size trade-off.
To further shrink the \spx signature size, the top $z$ layers of the
hypertree can be merged together into a a single tree of height $zh'$. That way
an \spx signature includes $z-1$ less \wotsp signatures. This decreases the
signature size by $n\cdot len(z-1)$ bytes, but typically comes at
the cost of speed as now a tree of height $zh'$ has to be computed for
each signature generation.
%For $z=2$ the signature shrinks by 1kb for w=256 and 2kb for w=16.
This can be prevented by storing the
%
% The trade-off of that optimization is that the intermediate tree significantly increases the signing
% time as it would take more time calculate the bigger intermediate tree in order to build the
% authentication path to the root for the signature. To alleviate the extra cost, the
nodes at height $ih'$, where $0 < i < z$, %would need to be kept
as part of the secret key. These nodes (auxiliary data) can
be used to build the authentication paths to the root of the merged tree without
actually computing the whole tree. Indeed, authentication path computation in
this case gets faster than computing the authentication paths for $z$ tree layers
in the original hypertree.
The size of the auxiliary data
is %$n2^{ih'}$. %$
%
% Andy: @Panos the simple formula without sum only applies for z=2
%       which we did not establish yet.
%
$n \sum_{i=1}^{z-1} 2^{ih'}$. %For \spx, this optimization is used with $z=2$. One more significant trade-off is
While this already grows extremely fast, the real problem turned out to be
key generation time. As the full tree still has to be
computed once during key generation, key generation time increases.
Key generation would now take
$2^{zh'}$ \wotsp key generations.

Initial experiments suggested that key generation time easily moves into
the order of minutes already for $z=2$ while the benefit in signature size is
1KB or 2KB for $w=256$ and $w=16$ respectively. In addition, this optimization
significantly complicates implementations as the top tree has to be handled
differently than the remaining trees. Hence, this idea was discarded.
% \todo{Optionally, put in a drawing that shows how the $z-1$ trees are merged, there are no WOTS+ signatures any more and what the auxiliary nodes kept are. }
