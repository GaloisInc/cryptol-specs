% @author Daniel J. Bernstein
% @author Andreas Hülsing
% @author Stefan Kölbl
% @author Ruben Niederhagen
% @author Joost Rijneveld
% @author Peter Schwabe
% @copyright The authors
%
% Within this file, the `code` blocks have the following copyright:
% @copyright Galois, Inc.
% @author Marios Georgiou <marios@galois.com>
%\section{The \spx Construction}
\begin{code}
  module sphincs_spec where
\end{code}
\label{sec:spx}
We now have all ingredients to describe our main construction \spx.
Essentially, \spx is an orchestration of the methods and schemes described
before. It only adds randomized message compression and verifiable index
generation.

\subsection{\spx Parameters}

\spx has the following parameters:
\begin{description}
 \item  $n$ : the security parameter in bytes.
 \item  $w$ : the Winternitz parameter as defined in \autoref{sec:wots:params}.
 \item  $h$ : the height of the hypertree as defined in \autoref{sec:ht:params}.
 \item  $d$ : the number of layers in the hypertree as defined in \autoref{sec:ht:params}.
 \item  $k$ : the number of trees in \fors as defined in \autoref{sec:fors:params}.
 \item  $t$ : the number of leaves of a \fors tree as defined in \autoref{sec:fors:params}.
\end{description}

All the restrictions stated in the previous sections apply. Recall that
we use $a = \log t$. Moreover, from these values the values $m$ and \len are
computed as
\begin{itemize}
  \item $m$: the message digest length in bytes.
  It is computed as
  $$m=\floor{(k\log t +7)/ 8} + \floor{(h - h/d +7)/ 8} + \floor{(h / d +7)/ 8}.$$

  While only $h + k\log t$ bits would be needed, using the longer $m$ as defined
  above simplifies implementations significantly.
  \item $\len$: the number of $n$-byte string elements in a \wotsp private
        key, public key, and signature. It is computed as $\len =
        \len_1 + \len_2$, with
        \begin{equation*}
          \len_1 = \ceil*{\frac{8n}{\log(w)}},\
          \len_2 = \floor*{\frac{\log{(\len_1(w - 1))}}{\log(w)}} + 1
        \end{equation*}
\end{itemize}

In the following, we assume that all algorithms have access to these parameters.

\subsection{\spx Key Generation (Function \spxkgen)}

   The \spx private key contains two elements. First, the $n$-byte secret seed
   \sseed which is used to generate all the \wotsp and \fors private key elements.
   Second, an $n$-byte PRF key \skprf which is used to deterministically
   generate a randomization value for the randomized message hash.

   The \spx public key also contains two elements. First, the \hyper public key,
   i.e. the root of the tree on the top layer. Second, an $n$-byte public seed
   value \pseed which is sampled uniformly at random.

   As \spxsign does not get the public key, but needs access to \pseed (and
   possibly to \proot for fault attack mitigation), the \spx secret key contains
   a copy of the public key.

   The description of algorithm \spxkgen assumes the existence of a function
   \texttt{sec\_rand} which on input $i$ returns $i$-bytes of cryptographically strong
   randomness.

\begin{lstlisting}[label=alg:spx:pkgen, language=pseudoc,
                   caption=\spxkgen\ -- Generate a \spx key pair.]

# Input: (none)
# Output: SPHINCS+ key pair (SK,PK)

spx_keygen( ){
     SK.seed = sec_rand(n);
     SK.prf = sec_rand(n);
     PK.seed = sec_rand(n);
     PK.root = ht_PKgen(SK.seed, PK.seed);
     return ( (SK.seed, SK.prf, PK.seed, PK.root), (PK.seed, PK.root) );
}

\end{lstlisting}

    The format of a \spx private and public key is given in \autoref{fig:spx:keys}.

\begin{figure} [h]
  \begin{center}
  \begin{minipage}{.4\textwidth}
        \begin{center}
	  \begin{tabular}{|c|}
	    \hline
	    \\[-0.5em] \sseed ($n$ bytes) \\[-0.5em] \\ \hline
	    \\[-0.5em] \skprf ($n$ bytes) \\[-0.5em] \\ \hline
	    \\[-0.5em] \pseed ($n$ bytes) \\[-0.5em] \\ \hline
	    \\[-0.5em] \proot ($n$ bytes) \\[-0.5em] \\ \hline
	  \end{tabular}
        \end{center}
      \end{minipage}
  \begin{minipage}{.4\textwidth}
      \begin{center}
	\begin{tabular}{|c|}
	    \hline
	    \\[-0.5em] \pseed ($n$ bytes) \\[-0.5em] \\ \hline
	    \\[-0.5em] \proot ($n$ bytes) \\[-0.5em] \\ \hline
	  \end{tabular}
        \end{center}
      \end{minipage}
  \end{center}
  \caption{Left: \spx secret key. Right: \spx public key.}
  \label{fig:spx:keys}
\end{figure}

\subsection{\spx Signature}

   A \spx signature \htsig is a byte string of length $(1 + k(a + 1) + h + d\len)n$.
   It consists of an $n$-byte randomization string $R$, a \fors signature
   \forssig consisting of $k(a+1)$ $n$-byte strings, and a \hyper signature \htsig
   of $(h + d\len)n$ bytes.

   The data format for a signature is given in \autoref{fig:xmssmt:sig}

%Not so elegant way for the table in the figure below with the right row height.
\begin{figure} [h]
  \begin{center}
    \begin{tabular}{|c|}
      \hline
      \\[-0.5em] Randomness \Random ($n$ bytes) \\[-0.5em] \\ \hline
      \\[-0.5em] \fors signature \forssig ($k(a+1) \cdot n$ bytes) \\[-0.5em] \\ \hline
      \\[-0.5em] \hyper signature \htsig ($(h +d\len)n$ bytes) \\[-0.5em] \\ \hline
    \end{tabular}
  \end{center}
  \caption{\spx signature}
  \label{fig:spx:sig}
\end{figure}

\subsection{\spx Signature Generation (Function \spxsign)}
   Generating a \spx signature consists of four steps. First, a random value
   \Random
   is pseudorandomly generated. Next, this is used to compute a $m$ byte message
   digest which is split into a $\floor{(k\log t +7)/ 8}$-byte partial message
   digest $\texttt{tmp\_md}$, a $\floor{(h - h/d +7)/ 8}$-byte tree index $\texttt{tmp\_idx\_tree}$, and
   a $\floor{(h / d +7)/ 8}$-byte
   leaf index $\texttt{tmp\_idx\_leaf}$. Next, the actual values \md,
   $\texttt{idx\_tree}$, and $\texttt{idx\_leaf}$ are computed by extracting the
   necessary number of bits.
   The partial message digest
   \md is then signed with the $\texttt{idx\_leaf}$-th
   \fors key pair of the $\texttt{idx\_tree}$-th \xmss tree on the lowest \hyper layer.
   The public key of the \fors key pair is then signed using
   \hyper. As described in Section~\ref{sec:hyper:sign}, the index is never
   actually used as a whole, but immediately split into a tree index
   and a leaf index, for ease of implementation.

   When computing \Random, the PRF takes a $n$-byte string \texttt{opt}
   which is initialized with \pseed but can be overwritten with randomness
   if the global variable RANDOMIZE is set. This option is given as otherwise
   \spx signatures would be always deterministic. This might be problematic in
   some settings. See \autoref{sec:security} and \autoref{sec:discussion} for
   more details.

\begin{lstlisting}[label=alg:spx:sign, mathescape, language=pseudoc,
                   caption=\spxsign\ -- Generating a \spx signature]

# Input: Message M, private key SK = (SK.seed, SK.prf, PK.seed, PK.root)
# Output: SPHINCS+ signature SIG

spx_sign(M, SK){
     // init
     ADRS = toByte(0, 32);

     // generate randomizer
     opt = PK.seed;
     if(RANDOMIZE){
       opt = rand(n);
     }
     R = PRF_msg(SK.prf, opt, M);
     SIG = SIG || R;

     // compute message digest and index
     digest = H_msg(R, PK.seed, PK.root, M);
     tmp_md = first floor((ka +7)/ 8) bytes of digest;
     tmp_idx_tree = next floor((h - h/d +7)/ 8) bytes of digest;
     tmp_idx_leaf = next floor((h/d +7)/ 8) bytes of digest;

     md = first ka bits of tmp_md;
     idx_tree = first h - h/d bits of tmp_idx_tree;
     idx_leaf = first h/d bits of tmp_idx_leaf;

     // FORS sign
     ADRS.setLayerAddress(0);
     ADRS.setTreeAddress(idx_tree);
     ADRS.setType(FORS_TREE);
     ADRS.setKeyPairAddress(idx_leaf);

     SIG_FORS = fors_sign(md, SK.seed, PK.seed, ADRS);
     SIG = SIG || SIG_FORS;

     // get FORS public key
     PK_FORS = fors_pkFromSig(SIG_FORS, md, PK.seed, ADRS);

     // sign FORS public key with HT
     ADRS.setType(TREE);
     SIG_HT = ht_sign(PK_FORS, SK.seed, PK.seed, idx_tree, idx_leaf);
     SIG = SIG || SIG_HT;

     return SIG;
}
\end{lstlisting}

\subsection{\spx Signature Verification (Function \spxverify)}

   \spx signature verification (\autoref{alg:spx:ver}) can be summarized as
   recomputing message digest and index, computing a candidate \fors public key,
   and verifying the \hyper signature on that public key. Note that the \hyper
   signature verification will fail if the \fors public key is not matching the
   real one (with overwhelming probability). \spx signature verification takes
   a message \msg, a signature \spxsig, and a \spx public key \PK.

\begin{lstlisting}[breaklines=true, label=alg:spx:ver, mathescape, language=pseudoc,
                   caption=\spxverify\ -- Verify a \spx signature \spxsig on a
   message \msg using a \spx public key \PK]

# Input: Message M, signature SIG, public key PK
# Output: Boolean

spx_verify(M, SIG, PK){
     // init
     ADRS = toByte(0, 32);
     R = SIG.getR();
     SIG_FORS = SIG.getSIG_FORS();
     SIG_HT = SIG.getSIG_HT();


     // compute message digest and index
     digest = H_msg(R, PK.seed, PK.root, M);
     tmp_md = first floor((ka +7)/ 8) bytes of digest;
     tmp_idx_tree = next floor((h - h/d +7)/ 8) bytes of digest;
     tmp_idx_leaf = next floor((h/d +7)/ 8) bytes of digest;

     md = first ka bits of tmp_md;
     idx_tree = first h - h/d bits of tmp_idx_tree;
     idx_leaf = first h/d bits of tmp_idx_leaf;

     // compute FORS public key
     ADRS.setLayerAddress(0);
     ADRS.setTreeAddress(idx_tree);
     ADRS.setType(FORS_TREE);
     ADRS.setKeyPairAddress(idx_leaf);

     PK_FORS = fors_pkFromSig(SIG_FORS, md, PK.seed, ADRS);

     // verify HT signature
     ADRS.setType(TREE);
     return ht_verify(PK_FORS, SIG_HT, PK.seed, idx_tree, idx_leaf, PK.root);
}
\end{lstlisting}

